\chapter{Rational Choice}

\section{Präferenzen}

Ein \textbf{Güterbündel} ist eine vollständige Liste mit Mengenangaben für alle verfügbaren Güter. Mittels einer sogenannten \textbf{Präferenzordnung} können Güterbündel verglichen und damit Präferenzen ausdrückt werden. ~\bigskip

Für beliebige Güterbündel $x$ und $y$ führen wir die folgende Notation ein:
\begin{itemize}
	\item falls $x$ gegenüber $y$ strikt präferiert wird, dann $x \succ y$
	\item falls $x$ gegenüber $y$ schwach präferiert wird, dann $x \succeq y$
	\item falls der Konsument zwischen $x$ und $y$ indifferent ist, dann $x \sim y$
\end{itemize}

Präferenzordnungen sind allerdings nur ordinal und nicht kardinal. Wir können also nicht aussagen, \enquote{um wieviel besser} ein Bündel gegenüber einem anderen ist.

\subsubsection*{Die Axiome für Präferenzrelationen}

Folgende Eigenschaften können auf eine Präferenzordnung zutreffen:

\begin{itemize}
	\item \textbf{Vollständigkeit}: für jedes Paar verfügbarer Güterbündel $x$ und $y$ gilt entweder $x \succeq y$ oder $y \succeq y$. 
	\item \textbf{Reflexivität}: (dies folgt bereits aus Vollständigkeit) bei der Wahl zwischen zwei identischen Güterbündeln liegt keine strikte Präferenz vor: $x \succeq x$.
	\item \textbf{Transitivität}: für jede beliebige Auswahl von 3 Bündeln $x$, $y$ und $z$ soll gelten, dass aus $x \succeq y$ und $y \succeq z$ folgt, dass $x \succeq z$.
\end{itemize}

Eine Präferenzordnung die vollständig und transitiv ist, nennen wir Präferenzrelation. Häufig werden aber auch zusätzliche Eigenschaften gefordert:
\begin{itemize}
	\item \textbf{Strenge Monotonie}: Eine Präferenzrelation $\succeq$ ist streng monoton genau dann, wenn für alle Konsumpläne $x$ und $y$ gilt: wenn $x$ von jedem Gut mindestens so viel enthält wie $y$ und wenn $x$ von mindestens einem Gut echt mehr enthält als $y$, dann gilt $x \succ y$. 
	\item \textbf{Strenge Monotonie}: Eine Präferenzrelation $\succeq$ ist monoton genau dann, wenn für alle Konsumpläne $x$ und $y$ gilt: wenn $x$ von jedem Gut echt mehr enthält als $y$, dann gilt $x \succ y$. 
	\item \textbf{(Streng) konvexe Präferenzen}: Mischungen von indifferenten Konsumplänen sind (echt) bessre, als die der Mischung zugrundeliegenden Konsumpläne. Formal: wenn $x \tilde y$, dann gilt $\lambda x + (1 - \lambda) y \succeq (\succ) x$ für alle $\lambda \in [0, 1]$ ($\lambda \in (0, 1)$).
\end{itemize}

\subsubsection*{Indifferenzmenge}

Alle Konsumpläne, die nach einer Präferenzordnung indifferent (\enquote{genauso gut}) zu einem gegebenen Konsumplan $x$ sind, werden in der sogenannten Indifferenzmenge zu $x$ zusammengefasst:
$$ I(x) = \big\{ y \in X ~|~x \sim y \big\} $$
\begin{itemize}
	\item Wenn die Präferenzen transitiv sind, können sich zwei Indifferenzkurven nie schneiden.
	\item Jede Präferenzordnung lässt sich als ein System von Indifferenzkurven darstellen.
	\item Ist die Präferenzrelation streng monoton, so hat $I(x)$ eine negative Steigung
	\item Ist die Präferenzrelation streng monoton, so liegen \enquote{bessere Konsumpläne} auf höheren Indifferenzkurven, \enquote{schlechtere} auf niedrigeren.
\end{itemize}

Die Annahme die im folgenden stets getroffen wird, dass es nämlich lediglich zwei Güter gibt, ist keine richtige Einschränkung, da Gut 2 stets \enquote{alle anderen Güter} zusammenfassen kann.

\section{Theorie des Haushalts}

 Wir benötigen für die folgende Theorie diese Begriffe:

\begin{itemize}
	\item \textbf{Budgetbeschränkung}: die Ausgaben dürfen das Einkommen nicht überschreiten 
		$$ p_1 \cdot x_1 + p_2 \cdot x_2 \leq m. $$
	\item \textbf{Budgetmenge}: die Menge aller Konsumpläne die ein Konsument sich leisten kann 
		$$ B(m, p_1, p_2) = \big\{ (x_1, x_2) \in \mathbb{R}^2 ~|~p_1 \cdot x_1 + p_2 \cdot x_2 \leq m \big\}. $$
	\item \textbf{Budgetgerade}: oberer Rand der Budgetmenge
		$$ p_1 \cdot x_1 + p_2 \cdot x_2 = m $$ 
\end{itemize}

Hier sei angemerkt, dass $- \frac{p_1}{p_2}$ die Steigung der Budgetgerade ist. ~\bigskip

\subsubsection*{Komparative Statik der Budgetmenge}

Die Budgetmenge kann sich durch verschiedene Einflüsse ändern. Darunter fallen:
\begin{itemize}
	\item Einkommensveränderung, Pauschalsteuern und Pauschalsubventionen: führt zu einer Parallelverschiebung der Budgetgerade
	\item Veränderung des Preisverhältnisses $- \frac{p_1}{p_2}$ durch z.B. Mengen- oder Wertsteuern und Mengen- oder Wertsubventionen: führt zu einer Änderung der Steigung der Budgetgerade
	\item Rationierung: schneidet die Budgetmenge ab einer Menge ab
	\item Gleichmäßige Wertsteuer: ändern sich alle Preise im gleichen Verhältnis ist dies äquivalent zu einer Einkommensbesteuerung; somit verschiebt sich die Budgetgerade auch hier parallel
\end{itemize} 

% todo Bild Budgetmengen und Verschiebungen der BM

\subsubsection*{Unterschiedliche Präferenztypen}

\begin{itemize}
	\item \textbf{Perfekte Substitute}: bei perfekten Substituten ist der Konsum bezüglich jeder Mischung zweier indifferenter Konsumplänen wiederum indifferent. Die Indifferenzkurven sind im 2-Güter-Fall fallende Gerade. Sie sind streng monoton und konvex. Bsp.: Wasser und Säfte.
	\item \textbf{Perfekte Komplemente}: bei perfekten Komplementen möchte ein Konsument immer in einem bestimmten Verhältnis konsumieren (1:1 oder 3:2). Die Indifferenzkurven sind im 2-Güter-Fall L-förmig, sowie monoton und konvex. Die \enquote{Ecken} geben das optimale Konsumverhältnis an. Beispiel: rechte und linke Schuhe oder eine Tasse Tee und zwei Löffel Zucker.
	\item \textbf{Präferenzen mit Sättigungspunkt}: ein Sättigungspunkt (bliss point) ist ein Konsumplan, den der Konsument allen anderen Konsumplänen vorzieht. Dies verletzt das Axiom der Monotonie! Beispiel: Bier und Brezeln.
	\item \textbf{Neutrale Güter}: ein Gut ist für einen Konsumenten ein neutrales Gut, wenn seie Menge in einem Konsumklan keinen Einfluss auf die Beurteilung durch den Konsumenten hat. Die Indifferenzkurven sind parallel zur Achse des anderen Gutes.
	\item \textbf{Goods/Bads}: durch Erhöhen eines Goods wird das neue Güterbündel dem ursprünglichen gegenüber präferiert. Durch Erhöhen eines Bads wird das ursprünglichen Güterbündel dem neuen gegenüber präferiert. 
\end{itemize}

\section{Nutzentheorie}

Eine Nutzenfunktion $u(x)$ ordnet jedem Güterbündel einen Nutzen zu. Genau wie die Präferenzrelationen so sind Nutzenfunktionen lediglich ein ordinales Konzept. Ein Güterbündel $x$ wird genau dann $y$ bevorzugt, wenn $u(x) \geq u(y)$.

\subsubsection*{Repräsentationssatz}

Gegeben sei eine Präferenzrelation die vollständig, transitiv und stetig ist (d.h. jede stetige Präferenzrelation), so existiert eine Funktion, die sie repräsentiert. Ist andersherum eine Nutzenfunktion gegeben, die eine Präferenzrelation repräsentiert, so ist sie vollständig und transitiv.

\subsubsection*{Grenzrate der Substitution (MRS)}

Die Grenzrate der Substitution gibt das Tauschverhältnis an, zu dem ein Konsument Gut 1 gegeben Gut 2 tauschen würde. Dies entsprecht der Steigung der Indifferenzkurve.
	$$ MRS = - \frac{MU_1}{MU_2} = - \frac{\partial u(x_1, x_2)}{\partial x_1} \big/ \frac{\partial u(x_1, x_2)}{\partial x_2} $$ % todo nutzen pro Preis muss gleich sein
$MU_1$ bzw. $MU_2$ beschreiben den Grenznutzen von Gut 1 bzw. Gut 2 und sowohl die, als auch die MRS ist ein lokales Konzept.

\subsubsection*{Positiv monotone Transformation}

Da Nutzenfunktionen ein lediglich ordinales Konzept sind, können wir in gewissem Maß deren Wert verändern. Eine positiv monotone Transformation ist eine Funktion $v(x_1, x_2)$, falls
$$ v(x_1, x_2) = f(u(x_1, x_2)), $$
mit f' > 0. Typische Beispiele sind
\begin{itemize}
	\item Addition einer Konstanten: $v(x_1, x_2) = u(x_1, x_2) + c, c \in \mathbb{R}$
	\item Multiplikation mit einem positiven Faktor: $v(x_1, x_2) = a \cdot u(x_1, x_2), a \in \mathbb{R}^+$
	\item Positive Exponenten: $v(x_1, x_2) = \left(u(x_1, x_2) \right)^{z}, z \in \mathbb{R}^+$
\end{itemize}

Jede positiv monotone Transformation einer Nutzenfunktion stellt dieselbe Präferenzordnung dar wie die ursprüngliche Nutzenfunktion. Damit ändern positiv monotone Transformationen auch die MRS nicht (allerdings die Grenznutzen $MU_i$)!

\subsubsection*{Nutzenfunktion für bekannte Präferenzordnungen}
\begin{itemize}
	\item \textbf{Perfekte Substitute}: $u(x_1, x_2) = a \cdot x_1 + b \cdot x_2$, $a, b > 0$. Es gilt $MRS = - \frac{a}{b}$ und damit sind die Indifferenzkurven fallende Geraden.
	\item \textbf{Perfekte Komplemente}: $u(x_1, x_2) = \min \{ a \cdot x_1, b \cdot x_2 \}$, $a, b > 0$. Der Konsument möchte hier im festen Verhältnis $b : a$ konsumieren (in den \enquote{Ecken} der Indifferenzkurven). Die Optimalitätsbedingung ist damit: $a \cdot x_1 = b \cdot x_2$.
	\item \textbf{Cobb-Doublas Nutzenfunktion}: $u(x_1, x_2) = x_1^c \cdot x_2^d$, $c, d > 0$. Dies kann (und sollte) positiv monoton transformiert werden: sei $\alpha = \frac{c}{c + d}$
		$$ u(x_1, x_2) = x_1^{\alpha} \cdot x_2^{1 - \alpha}, \quad \alpha \in (0, 1) $$
		Für $x_1, x_2 > 0$ repräsentieren Cobb-Douglas Nutzenfunktionen streng monotone, streng konvexe Präferenzen. Es gibt hier immer eine optimale Entscheidung in einer innere Lösung (siehe \enquote{optimale Entscheidung})
	\item \textbf{Quasilineare Präferenzen}: $u(x_1, x_2) = v(x_1) + x_2$ mit $v' > 0$, $v'' < 0$. Die Indifferenzkurven sind \enquote{quasi parallel verschobene} Versionen voneinander.
\end{itemize}

\section{Optimale Entscheidung und Nachfrage}

Durch die Budgetmenge können wir beschreiben, was der Agent sich leisten kann. Durch Präferenzen wird ausgedrückt, was der Agent konsumieren möchte. Im folgenden kombinieren wir beide Konzepte und treffen bei gegebenen Preisen und gegebenem Einkommen $m$ in Abhängigkeit von den individuellen Präferenzen des Agenten eine optimale Konsumentscheidung. ~\bigskip

Die zentrale Frage ist also: welcher Konsumplan in der Budgetmenge maximiert den Nutzen des Konsumenten?
	$$ \max_{x_1, x_2 \in \mathbb{R}^+} u(x_1, x_2) ~\text{ s.t. }~ p_1 \cdot x_1 + p_2 \cdot x_2 \leq m $$ 
	
Der \textbf{optimale Konsumplan} ist der Konsumplan, unter allen die in der Budgetmenge liegen, der auf der höchsten Indifferenzkurve liegt. Bei monotonen Präferenzen liegt der optimale Konsumplan immer auf der Budgetgerade; damit suchen wir die Indifferenzkurve die im Allgemeinen gerade die Budgetgerade berührt und es gilt im Allgemeinen auch:
		$$ MRS \overset{!}{=} - \frac{p_1}{p_2}, $$
d.h. die Steigung der Indifferenzkurve = Steigung der Budgetgerade.
\begin{itemize}
	\item \enquote{Im Allgemeinen} da wir bereits gesehen haben, dass die MRS nicht immer bestimmbar sein muss (z.B. bei perfekten Komplementen), nicht-konvexe Nutzenfunktionen gegeben sein können oder eine Randlösung existiert. Wir werden deswegen im Folgenden das Vorgehen für verschiedene Präferenzordnungen genauer betrachten.
	\item Intuitiv wird durch diese Bedingung gefordert, dass von beiden Gütern der gleiche Grenznutzen im Verhältnis zum Preis erbracht wird, da $\frac{MU_1}{p_1} = \frac{MU_2}{p_2}$.
\end{itemize}

\begin{kr}[Optimalität] ~\
	\begin{itemize}
		\item \textbf{Perfekte Komplemente}: $u(x_1, x_2) = \min \{ a \cdot x_1, b \cdot x_2 \}$. Das Optimum liegt immer an der Ecke der Indifferenzkurve, unabhängig vom Preisverhältnis. Die Nutzenfunktion gibt das Optimum $a \cdot x_1 = b \cdot x_2$ vor. Wir lösen diese Gleichung nach $x_1$ oder $x_2$ auf und setzen dies in die Budgetgerade ein.
		\item \textbf{Perfekte Substitute}: $u(x_1, x_2) = a \cdot x_1 + b \cdot x_2$. Die Steigung der Indifferenzkurve ist konstant und damit ist $MRS = - \frac{a}{b}$. Es können nun drei mögliche Fälle eintreten:
			\begin{itemize}
				\item die Budgetgerade verläuft steiler als die Indifferenzkurven: 
					$$ \big| - \frac{a}{b} \big| < \big| - \frac{p_1}{p_2} \big| $$
					$\Rightarrow$ Randlösung: es wird nur $x_2$ konsumiert, d.h. $x^* = (0, \frac{m}{p_2})$.
				\item die Budgetgerade verläuft flacher als die Indifferenzkurven: 
					$$ \big| - \frac{a}{b} \big| > \big| - \frac{p_1}{p_2} \big| $$
					$\Rightarrow$ Randlösung: es wird nur $x_1$ konsumiert, d.h. $x^* = (\frac{m}{p_1}, 0)$.
				\item Falls Budgetgerade und Indifferenzkurven dieselbe Steigung haben, d.h. 
					$$ \big| - \frac{a}{b} \big| = \big| - \frac{p_1}{p_2} \big| $$
					sind alle Güter auf der Budgetgerade optimal.
			\end{itemize}
		\item \textbf{Cobb-Douglas}: $u(x_1, x_2) = x_1^c \cdot x_2^d$, $c, d > 0$. Bei Cobb-Douglas Nutzenfunktionen gibt es immer eine innere Lösung. Man kann hier die Gleichung
			$$ MRS = - \frac{p_1}{p_2}, $$
			lösen und das Ergebnis in die Budgetgerade einsetzen. Der einfachere Weg ist allerdings die Nutzenfunktion positiv monoton zu transformieren, sodass mit $\alpha = \frac{c}{c + d}$
				$$ u(x_1, x_2) = x_1^{\alpha} \cdot x_2^{1 - \alpha}, \quad \alpha \in (0, 1) $$
			gilt. Das Optimum lässt sich dann direkt über die folgende Formel bestimmen:
			$$ x^* = \left( \frac{\alpha m}{p_1}, \frac{(1 - \alpha) m}{p_2} \right) $$
		\item \textbf{Quasilineare Präferenzen}: $u(x_1, x_2) = v(x_1) + x_2$ mit $v' > 0$, $v'' < 0$. Durch die besondere Form der Indifferenzkurven möchte man unabhängig vom Einkommen immer dieselbe Menge von Gut 1 konsumieren (falls die Nutzenfunktion linear in $x_2$ ist). Dies ist nur möglich, wenn das Einkommen $m$ dafür ausreicht und damit ei \enquote{kritisches Einkommen} $m^{krit}$ nicht unterschreitet:
			\begin{itemize}
				\item Bestimme $x_1^{krit}$ über die Optimalitätsbedingung $MRS = - \frac{p_1}{p_2}$
				\item Berechne kritisches Einkommen $m^{krit} = p_1 \cdot x_1^{krit}$. 
				\item Für $m > m^{krit}$: $x_1^* = x_1^{krit}$, $x_2^* = \frac{m - p_1 \cdot x_1^{krit}}{p_2}$
				\item Für $m \leq m^{krit}$: $x^* = (\frac{m}{p_1}, 0)$
			\end{itemize}
	\end{itemize}	
\end{kr}
