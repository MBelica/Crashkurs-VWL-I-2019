\chapter{Monopol}

Wenn die Angebotsentscheidung eines einzelnen Anbieters den Marktpreis beeinfluss, dann hat dieser Anbieter \textbf{Marktmacht}. Dies kann besonders dann der Fall sein, wenn wenige Anbieter auf dem Markt sind. ~\bigskip

Ist lediglich ein einziger Anbieter mit Marktmacht auf dem Markt, so nennt man ihn \textbf{Monopolist}. Er kann entweder den Preis oder die Menge festlegen, die jeweils andere Variable wird durch die gegebene Marktnachfrage bestimmt. Ein Monopolist kann dabei durch Verknappung des Angebots den Preis in die Höhe treiben, sodass ein Preis erreicht wird, der über den Grenzkosten liegt. Durch die mögliche Preiserhöhung in dieser Situation kann ein Wohlfahrtsverlust entstehen.

\subsubsection*{Gewinnmaximierungsproblem des Monopolisten}
$$\max \Pi(y) = R(y) - C(y) = p(y) \cdot y - C(y)$$
\begin{itemize}
	\item Im Monopol-Optimum gilt: $MR = MC$ (Grenzerlös = Grenzkosten). Erlös und Kostenfunktion haben also in diesem Punkt dieselbe Steigung.
	\item Wäre nämlich der zusätzliche Erlös (Grenzerlös) bei einer Erhöhung des Outputs größer als die zusätzlichen Kosten (Grenzkosten), dann könnte der Gewinn erhöht werden, indem man eine zusätzliche Outputeinheit herstellt. Analog für den anderen Fall.
	\item Ein gewinnmaximierender Monopolist wird immer im elastischen Bereich der Nachfrage $(|\epsilon_D(p)| \geq 1$) anbieten. Dies kann man unter gewissen Bedingungen aus der sogenannten Amoroso-Robinson-Gleichung ablesen: $p(y) \cdot \left[ 1 - \frac{1}{|\epsilon|} \right] = MC(y)$
	\item Die Höhe des monopolistischen Preisaufschlags hängt somit von der Preiselastizität der Nachfragefunktion ab: je elastischer die Nachfrage, umso geringer der Preisaufschlag. Im Fall einer unendlich elastischen Nachfrage sind Monopol- und Wettbewerbspreis gleich. 
\end{itemize} 

\begin{kr}[Monopolist]
	Um die optimale Ausbringungsmenge $y*$ zu ermitteln, lösen wir das Problem:
	$$ \max \Pi(y) = p(y) \cdot y - C(y) $$
	was zur folgenden Gleichung führt:
		$$ p(\hat{y}) + p'(\hat{y}) \cdot y = MC(\hat{y}) $$	
	Da der Monopolist die Option hat nicht zu produzieren, müssen wir überprüfen, ob sich dies lohnen würde. Ist $\Pi(\hat{y}) > \pi(0)$, so ist $y^* = \hat{y}$ und ansonsten $y^* = 0$.
\end{kr} ~\newpage

\subsubsection*{Monopolistischer Wettbewerb}

Sind mehrere Unternehmen im Markt etabliert und verfügt jede Unternehmung über Marktmacht bei gleichzeitiger Produktdifferenzierung, so spricht man von monopolistischen Wettbewerb. Jede Unternehmung verhält sich dabei als Monopolist; also ist die relevante Entscheidungsregel: 
$$ MR( y_1 ) = MC( y_1 ) $$
Die positiven Gewinne bilde einen Anreiz für Markteintritte, sodass sich die Nachfragekurve sich sukzessive nach innen verschiebt, was zu niedrigeren Preisen, niedrigerem individuellen Output und niedrigeren Gewinnen führt. Im langfristigen Gleichgewicht wird dann zu Durchschnittskosten produziert, die das Minimum der Durchschnittskosten überschreiten und dadurch entstehen Nullprofite, obwohl die Preise \enquote{zu hoch} sind und die Gleichgewichtsmenge zu niedrig. Die Effizienz und Wohlfahrt würde sich erhöhen wenn (weniger) Unternehmungen jeweils eine größere Menge produzieren würden, die gerade ihrer effizienten Betriebsgröße entspricht. Ein solcher monopolistischer Wettbewerb ist also ineffizient.