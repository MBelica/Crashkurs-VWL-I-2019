\chapter{Marktformen und vollkommener Wettbewerb}

Wenn neue Unternehmen in eine Branche eintreten, steigt das Angebot, wodurch die (inverse) Angebotskurve flacher wird. Daraus folgt, dass der Gleichgewichtspreis sinkt und sich der Gewinn verringert. Wenn andererseits Unternehmen aus dem Markt austreten, wird die (inverse) Angebotskurve steiler. ~\smallskip

Kurzfristig kann es sowohl zu ökonomischen Gewinnen oder ökonomischen Verlusten kommen. Sobald jedoch eine Firma Verluste schreibt, so muss sie sich zwischen einem Shutdown und einem Marktaustritt entscheiden.
\begin{itemize}
	\item Die Entscheidung die Produktion temporär einzustellen (\textbf{Shutdown}), $y = 0$ ist eine Entscheidung hinsichtlich der kurven Frist. Dies tritt ein, falls $p < AVC$.
	\item Falls der Preis langfristig unter den Durchschnittskosten bleibt, ist es für eine Unternehmung zur Vermeidung permanenter Verlust optimal, aus dem Markt auszutreten (\textbf{Marktaustritt}). Dies tritt ein, falls $p < AC$.
\end{itemize}

In einer positiven ökonomischen Gewinne  treten neue Unternehmen in den Markt ein, so dass die Angebotskurve flacher wird. Daraus folgt, dass der Gleichgewichtspreis sinkt und sich der Gewinn verringert.

\subsubsection*{Vollkommener Wettbewerb}

Eine Situation in der alle Anbieter als Preisnehmer agieren und lediglich über ihre Angebotsmenge entscheiden, nennt man vollkommener Wettbewerb (z.B. unbegrenzte Markteintritte). Jeder Anbieter maximiert dann seinen Gewinn bezüglich $y$:
	$$ \max_y \Pi(y) = R(y) - C(y) = p \cdot y - C(y) $$
	
\begin{itemize}
	\item \textbf{Marginal Revenue}: Der Grenzerlös $MR(y)$ ist die erste Ableitung der Erlösfunktion $R(y)$.
	\item \textbf{Marginal Costs}: Die Grenzkosten $MC(y)$ sind die erste Ableitung der Kostenfunktion $C(y)$
	\item Im Gewinnmaximum gilt nach Bedingung erster Ordnung demnach: Grenzerlös = Grenzkosten ($MR = MC$)
\end{itemize}

Ein Unternehmen bietet bei vollständiger Konkurrenz somit bei jedem Preis diejenige Menge an, bei der die Grenzkosten dem gegebenem Preis entsprechen. Dies ist die gewinnmaximierende Outputmenge.

\begin{kr}[Unternehmungen im vollkommenen Wettbewerb]
	Da bei vollkommendem Wettbewerb die Anbieter als Preisnehmer agieren, muss im Optimum gelten:
	$$ MC(y) = p(y) $$
	Auflösen dieser Gleichung nach $y$ ergibt die optimale Produktionsmenge.
\end{kr} % todo \subsubsection*{Langfristiges Marktgleichgewicht}

