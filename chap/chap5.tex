\chapter{Nachfrageanalyse}

Ist eine optimale Entscheidung getroffen, ist die nächstgelegene Frage, wie sich diese Entscheidung abhängig von Parametern ändert. ~\bigskip

Nachfrageänderung aufgrund einer Einkommensänderung:
\begin{itemize}
	\item \textbf{Normale Güter}: die nachgefragte Menge steigt nach einem Einkommensanstieg und fällt nach einem Einkommensabfall ($\frac{\Delta x_i}{\Delta m} > 0$).
	\item \textbf{Inferiore Güter}: die nachgefragte Menge sinkt nach einem Einkommensanstieg und steigt nach einem Einkommensabfall ($\frac{\Delta x_i}{\Delta m} < 0$).
\end{itemize}

Nachfrageänderung aufgrund der Preisänderung eines Gutes:
\begin{itemize}
	\item \textbf{Gewöhnliche Güter}: die nachgefragte Menge sinkt nach einer Preiserhöhung und steigt nach einer Preissenkung ($\frac{\Delta x_i}{\Delta p_i} < 0$).
	\item \textbf{Giffen Güter}: die nachgefragte Menge steigt nach einer Preiserhöhung und sinkt nach einer Preissenkung ($\frac{\Delta x_i}{\Delta p_i} > 0$).
\end{itemize}

\section{Komperative Statistik der Nachfrage}

Wie ändert sich die Nachfrage nach einem Gut aufgrund von Einkommens- und Preisänderungen.

\subsubsection*{Auswirkung einer Einkommensänderung auf die Nachfrage bei konstanten Preisen}

\begin{itemize}
	\item \textbf{Einkommensexpansionspfad (EEP)}
		\begin{itemize}
			\item Verbindung aller optimalen Konsumpläne in Abhängigkeit vom Einkommen $m$ im $(x_1, x_2)$-Diagramm.
			\item Grafisch bestimmt man den EEP durch Parallelverschiebung der Budgetgeraden.
		\end{itemize} % todo Zeichnung EEP für normale Güter
	\item \textbf{Engelkurve}
		\begin{itemize}
			\item Abbildung der Änderung der Nachfrage eines Gutes $x_i$ (bei Einkommensänderung) im $(x_i, m)$-Diagramm
			\item Die Änderung der nachgefragten Menge des anderen Gutes $x_j$ wird nicht explizit abgebildet.
		\end{itemize} % todo Zeichnungen
\end{itemize}

\subsubsection*{Auswirkung einer Eigenpreisänderung auf die Nachfrage des Gutes bei konstanten restlichen Preisen und gegebenem Einkommen}

\begin{itemize} \setcounter{enumi}{2}
	\item \textbf{Preisexpansionspfad (PEP)}
		\begin{itemize}
			\item Bildet ab, wie sich der optimale Konsumplan nach einer Preisänderung anpasst
			\item Grafisch im $(x_1, x_2)$-Diagramm durch Drehung der Budgetgerade im Punkt $\frac{m}{p_j}$.
		\end{itemize}
	\item \textbf{Nachfragekurve}
		\begin{itemize}
			\item Abbildung der Änderung der Änderung der Nachfrage eines Gutes $x_i$ im $(x_i, p_i)$-Diagramm
			\item Die Änderung der nachgefragten Menge des anderen Gutes $x_j$ wird nicht explizit abgebildet
		\end{itemize}	
\end{itemize}

EEP, Engelkurve, PEP und Nachfragekurve sehen für unterschiedliche Nachfragefunktionen bzw. Präferenzordnungen unterschiedlich aus. Wenn sich allerdings alle Preise und das Einkommen simultan um denselben Faktor ändern, ändert sich die Nachfrage nicht.

\subsubsection*{Auswirkung einer Fremdpreisänderung auf die Nachfrage eines Gutes bei konstantem Eigenpreis und gegebenem Einkommen}

\begin{itemize}
	\item \textbf{Substitute}: Güter bei denen, falls ein Gut teurer wird, sich der  Konsument bemüht, dieses Gut durch sein Substitut zu ersetzen ($\frac{\Delta x_i}{\Delta p_j} > 0$)
	\item \textbf{Komplemente}: Güter bei denen, falls ein Gut teurer wird, der Konsument weniger davon nachfragen wird und auch den Konsum der komplementären Güter einschränkt ($\frac{\Delta x_i}{\Delta p_j} < 0$)
\end{itemize}

\section{Stusky-Zerlegung}

Eine Veränderung des Preises eines Guts (z.B. $p_1$ sinkt auf $p_1'$) hat zwei Effekte auf die Nachfrage:
\begin{itemize}
	\item Die Budgetmenge wird größer, wodurch die Kaufkraft steigt. Die allein daraus resultierende Änderung der Nachfrage = Einkommenseffekt.
	\item Das Preisverhältnis $\frac{p_1}{p_2}$ ändert sich, wodurch die Budgetgerade flacher wird. Die hieraus  resultierende Änderung der Nachrage bei konstanter Kaufkraft = Substitutionseffekt
\end{itemize}

\subsubsection*{Gesamteffekt der Preisänderung}

Durch die Preisänderung ändert sich die Nachfrage, was wir im Gesamteffekt festhalten:
	$$ \Delta x_1^{GE} = x_1^{*}(p_1', p_2, m) - x_1^*(p_1, p_2, m) $$
	
\subsubsection*{Substitutionseffekt}

$$ \Delta x_1^{s} = x_1^*(p_1', p_2, m') - x_1^*(p_1, p_2, m) $$

\begin{itemize}
	\item Nachfrageänderung aufgrund Änderung des relativen Preisverhältnisses
	\item Kaufkraft wird konstant gehalten (altes optimales Güterbündel $x^*$ liegt auf der Budgetgeraden), wodurch wird ein hypothetisches Einkommen $m'$ einführen müssen:
		$$ m' = \Delta m + m ~\text{ mit } ~\Delta m = x_1^* \cdot \Delta p_1 = x_1^* \cdot (p_1' - p_1)  $$
	\item Die Nachfrageänderung aufgrund des Substitutionseffekts ist immer entgegengesetzt der auslösenden Preisänderung (auch bei Giffen-Gütern!)
\end{itemize}

\subsubsection*{Einkommenseffekt}

$$ \Delta x_1^{n} = x_1^{*}(p_1', p_2, m) - x_1^*(p_1', p_2, m') $$

\begin{itemize}
	\item Nachfrageänderung aufgrund der Änderung der Kaufkraft (relatives Preisverhältnis konstant)
\end{itemize}

\begin{kr}[Stusky-Zerlegung] ~\
	Zur Berechnung der Effekte benötigen wir:
	\begin{itemize}
		\item $x_1^*(p_1, p_2, m)$: optimales Bündel vor der Preisänderung
		\item $x_1^*(p_1', p_2, m')$: optimales Bündel bei neuen Preisen und fiktivem Einkommen $m'$
		\item $x_1^*(p_1', p_2, m)$: optimales Bündel nach Preisänderung
	\end{itemize}
	Dann ist
	\begin{itemize}
		\item Substitutionseffekt: $\Delta x_1^{s} = x_1^*(p_1', p_2, m') - x_1^*(p_1, p_2, m)$
		\item Einkommenseffekt: $\Delta x_1^{n} = x_1^{*}(p_1', p_2, m) - x_1^*(p_1', p_2, m')$
		\item Gesamteffekt: $\Delta x_1^{GE} = \Delta x_1^{s} + \Delta x_1^{n}$.
	\end{itemize}
\end{kr}
% todo Bild

% todo Hicks Diskutieren Sie kurz den Unterschied zwischen den Ideen, die der Hicks- und Slutsky-Zerlegung zugrunde liegen