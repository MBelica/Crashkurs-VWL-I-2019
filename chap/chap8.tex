\chapter{Marktformen und vollkommener Wettbewerb}

Produziert eine Firma auf einem beliebigen und ist ihr Gewinn negativ, so muss sie sich zwischen einem Shutdown und einem Marktaustritt entscheiden.
\begin{itemize}
	\item Die Entscheidung die Produktion temporär einzustellen (\textbf{Shutdown}), $y = 0$ ist eine Entscheidung hinsichtlich der kurven Frist. Dies tritt ein, falls $p < AVC$.
	\item Falls der Preis langfristig unter den Durchschnittskosten bleibt, ist es für eine Unternehmung zur Vermeidung permanenter Verlust optimal, aus dem Markt auszutreten (\textbf{Marktaustritt}). Dies tritt ein, falls $p < AC$.
\end{itemize} % todo kurze erklärung

Bei vollkommenem Wettbewerb sind alle Anbieter Preisnehmer und entscheiden lediglich über die Angebotsmenge. Jeder Anbieter maximiert seinen Gewinn bezüglich $y$:
	$$ \max_y \Pi(y) = R(y) - C(y) = p \cdot y - C(y) $$
	
\begin{itemize}
	\item \textbf{Marginal Revenue}: Der Grenzerlös $MR(y)$ ist die erste Ableitung der Erlösfunktion $R(y)$.
	\item \textbf{Marginal Costs}: Die Grenzkosten $MC(y)$ sind die erste Ableitung der Kostenfunktion $C(y)$
	\item Im Gewinnmaximum gilt nach Bedingung erster Ordnung demnach: Grenzerlös = Grenzkosten ($MR = MC$)
\end{itemize}

Ein Unternehmen bietet im vollständigen Wettbewerb somit bei jedem Preis diejenige Menge an, bei der die Grenzkosten dem gegebenem Preis entsprechen. Dies ist die gewinnmaximierende Outputmenge.

\begin{kr}[Unternehmungen im vollkommenen Wettbewerb]
	Da bei vollkommendem Wettbewerb die Anbieter als Preisnehmer agieren, muss im Optimum gelten:
	$$ MC(y) = p(y) $$
	Auflösen dieser Gleichung nach $y$ ergibt die optimale Produktionsmenge.
\end{kr}